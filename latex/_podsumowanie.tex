


\chapter*{Zakończenie}
 \addcontentsline{toc}{chapter}{Zakończenie}

\thispagestyle{empty}

\section*{Podsumowanie}
 \addcontentsline{toc}{section}{Podsumowanie}
Aplikacja została zrealizowana zgodnie z założeniami oraz spełnia cel pracy. Został zaimplementowany mechanizm do wykrywania wysokości tonu. Utworzony został również graficzny interfejs użytkownika, pozwalający na łatwe użytkowanie aplikacji. Do aplikacji dodano obsługę protokołu MIDI, dzięki której można odgrywać dźwięki na podstawie uzyskanych pomiarów wysokości tonów. Każdy załadowany plik dźwiękowy poddany analizie może również zostać zobrazowany w formie wykresu. Ponadto aplikacja została przetestowana na różnych systemach operacyjnych takich jak: Windows w wersji 7 Professional oraz Linux Minut 17.


Powstała aplikacja z pewnością może być wykorzystywany przez muzyków. Jako otwarte oprogramowanie może również stanowić bazę do dalszego rozwoju podobnych aplikacji. Dokładność aplikacji jest zdecydowanie zadowalająca.




 
\section*{Dalsze prace}
 \addcontentsline{toc}{section}{Dalsze prace}
Projekt stworzony w ramach niniejszej pracy inżynierskiej jest ukończony, jednak są to jedynie podstawowe funkcje w porównaniu do komercyjnych odpowiedników. Aplikację można również rozbudować o bardziej precyzyjne ustawienia protokołu MIDI tj. barwę odgrywanego dźwięku lub np. rodzaj wirtualnego instrumentu. Następnym zadaniem może być również zaimplementowanie mechanizmu do wykrywania wysokości kilku tonów jednocześnie. Kolejnym zadaniem, które znacznie wpłynie na atrakcyjność produktu, jest implementacja mechanizmu rejestrującego odgrywane dźwięki oraz zapisu do formatu MIDI. Umożliwiłoby to automatyczny zapis odgrywanego dźwięku na rzeczywistym instrumencie do zapisu nutowego. Funkcjonalność ta znacznie zautomatyzowałaby procesy konwersji utworów muzycznych do zapisu nutowego. Po utworzeniu takiej funkcjonalności możliwe byłoby przesyłanie drogą elektroniczną odgrywanych dźwięków oraz zapisu ich i magazynowania na dysku.


Aplikacja spełniła oczekiwane zadanie i działa w sposób poprawny i zgodny z zamierzonym w początkowych etapach projektowania aplikacji. Algorytmy, które zostały wykorzystane, również potwierdziły swoją skuteczność w wykrywaniu wysokości tonu w czasie rzeczywistym.