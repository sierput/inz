


\begin{titlepage}

 \begin{center}

  \vfill                          % do pionowego wyśrodkowania na stronie
    { 
        Z A C H O D N I O P O M O R S K I\\ U N I W E R S Y T E T
        \hspace{0.3 cm}
        T E C H N O L O G I C Z N Y\\ W
        \hspace{0.3 cm}
        S Z C Z E C I N I E\\
        \vspace{0.7 cm}	
        {  \textbf{WYDZIAŁ INFORMATYKI} }
    } \\
    \vspace{1.cm}
    \begin{figure}[h!]
    \centerline{\includegraphics[width=0.2\linewidth, natwidth=279,natheight=480]{rys/wizut_logo.png}} % godło
	\end{figure}

    \vspace{1.cm}
      { \Large {Mateusz Edward Sierputowski} \\ } 
      
    \vspace{0.2 cm}
      {Kierunek Informatyka} \\

  \vspace{1.cm}                   % odstęp pionowy 1 cm
  
     { \huge \textbf{Implementacja mechanizmów ekstrakcji tonów z sygnału akustycznego.} \\
       \huge \textbf{} \\
     } 

\end{center}

\begin{flushleft}
  \vspace{1.0 cm}
  {
        \hspace{8.0 cm}    Praca dyplomowa inżynierska \\
        \hspace{8.0 cm}    napisana pod kierunkiem \\
        \hspace{8.0 cm}    \textbf{dr inż. Tomasza Mąki} \\
        \hspace{8.0 cm}    w Katedrze Architektury Komputerów i Teleinformatyki \\
  }

\end{flushleft}

\begin{center}
   \vspace{1.0 cm}
   {\large Szczecin, {2018}}
   \vfill                         % do pionowego wyśrodkowania na stronie
\end{center}

\end{titlepage}

\pagebreak

\begin{titlepage}

\begin{center}
\vspace{2.cm}                   
\LARGE {An implementation of schemes for tones extraction from audio signal.} \\ 
\vspace{1cm}                   
\large {Mateusz Edward Sierputowski}\\
\large Supervisor: { dr inż. Tomasz Mąka}\\
\vspace{1cm}
\end{center}

\textbf{Abstract}
\vspace{12pt}
 
The purpose of the work can be divided into two parts. The first of these is the implementation of an application that implements the mechanisms of tone extraction from an acoustic signal. The second part is the implementation of the MIDI controller support mechanism. For the purpose of this work, the Linux operating system and its programming interfaces will be used. One of the basic functions of the first part of the program will be a thorough analysis and preparation of information about the acquired sound. The second part of the application will expand the functionality of the application by interpreting data from the first part and using the MIDI interface so that you can easily control eg: MIDI controller, synthesizer or other devices based on the MIDI protocol.

\end{titlepage}

\pagebreak

\begin{center}
\large
\vspace{3 cm}
OŚWIADCZENIE 
\end{center} 
 
Oświadczam, że przedkładaną pracę inżynierską kończącą studia napisałem samodzielnie.
Oznacza to, że przy pisaniu pracy poza niezbędnymi konsultacjami, nie korzystałem z
pomocy innych osób, a w szczególności nie zlecałem opracowania rozprawy lub jej części 
innym osobom, ani nie odpisywałem rozprawy lub jej części od innych osób. Potwierdzam też
zgodność wersji papierowej i elektronicznej złożonej pracy. Mam świadomość, że poświadczenie nieprawdy
będzie w tym przypadku skutkowało cofnięciem decyzji o wydaniu dyplomu.  


\thispagestyle {empty}