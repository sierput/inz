\chapter*{Wstęp}
 \addcontentsline{toc}{chapter}{Wstęp}
\label{wstęp}
\thispagestyle{empty}

Analiza widmowa dźwięku w ostatnich latach wraz z rozwojem elektroniki oraz oprogramowania znalazła wiele zastosowań. Służy ona do pozyskiwania informacji na temat sygnału dźwiękowego. Rozwiązania z tej dziedziny nauki oraz możliwości które stwarza, cieszą się dużym zainteresowaniem. Precyzja analizy dźwięku oraz szybkość jej wykonywania jest na tyle duża, że można za jej pomocą sterować układami elektronicznymi w czasie rzeczywistym.

	W niniejszej pracy podjęto się próby stworzenia programu komputerowego, w którym zostaną zaimplementowane algorytmy do estymacji tonu. Uzyskane wyniki zostaną wykorzystane do sterowania kontrolerem MIDI. Aplikacja z założeń projektowych będzie przeznaczona na aktualnie dominujące systemy operacyjne. Aplikacja powinna być podzielona na dwie części. Pierwsza z nich będzie mieć na celu estymacje tonu muzycznego oraz przechowywanie informacji. Następna część powinna implementować mechanizm sterowania kontrolerem MIDI za pomocą protokołu o tej samej nazwie.
	
Pierwszy rozdział przedstawia cel pracy inżynierskiej oraz zakres prac. Kolejny rozdział zawiera przegląd istniejących rozwiązań oraz porównanie dostępnych aplikacji o funkcjonalności nawiązującej do tematyki pracy. W rozdziale czwartym znajduje się opis implementacji programowej i opis działania aplikacji. Rozdział piąty poświęcony jest badaniom eksperymentalnym oraz prezentacji i analizie wyników.