
\textbf{Opis dokumentu}
\normalsize

\footnotesize
\textbf{Uwagi dotyczące edycji pracy dyplomowej}\\

\textbf{\textcolor{red}{
\begin{itemize}
\item{Ta strona powinna być usunięta z ostatecznej wersji pracy dyplomowej!!!,}\\
\item{Wszystkie teksty pisane w szablonie \rdm{kolorem różnym od czarnego}, należy zastąpić własnymi określeniami i opisami (zmieniając czcionkę na kolor czarny)},
\item{Proszę nie zmieniać struktury pracy prezentowanej w tym szablonie i starać się jak najlepiej dostosować treść do tej struktury.}\\
\end{itemize}}}

Podczas pisania pracy proszę stosować się do poniższych reguł:
\begin{itemize}
\item{pisząc pracę trzeba na bieżąco dbać o jej poprawne formatowanie (poprawność pisowni, znaki interpunkcyjne, unikanie wychodzenia na margines, poprawne rozmieszczenie rysunków, pisownia małą/wielką literą, spacje we właściwych miejscach, itp.), proszę \textcolor{red}{NIGDY} nie stosować zasady: "teraz napiszę byle jak, a później poprawię",}
\item{każde zdanie pracy powinno być przemyślane i logiczne, nie należy pisać tekstu nie niosącego istotnych informacji, tekst pracy powinien być jak najkrótszy, ale z zachowaniem zakresu przekazywanych informacji, proszę unikać zdań wielokrotnie złożonych,}
\item{nie można stosować określeń kolokwialnych,
prace należy pisać bezosobowo, w stronie biernej (np. "Zaciągam pliki z serwera z gigantyczną szybkością", powinno być: "Pliką są transferowane z serwera z dużą prędkością"),}
\item{\textcolor{red}{każdy} element wzoru matematycznego powinien być ściśle zdefiniowany,}
\item{proszę stosować akapity oddzielające opisy różnych zagadnień (linia przerwy pomiędzy tekstem),}
\item{w tekście pracy powinno znaleźć się odwołanie do \textcolor{red}{każdego} rysunku i każdej tabeli umieszczonej w pracy (np. Na Rys.~\ref{fig:schemat} znajduje się ...),}
\item{tytuły rozdziałów/podrozdziałów powinny jak najlepiej charakteryzować treść rozdziału, proszę nie bać się rozbudowanych tytułów,}
\item{proszę zwracać uwagę na poprawność pisowni (ortografia, interpunkcja, itp.), tekst trzeba na bieżąco sprawdzać narzędziem do kontroli pisowni.}
\end{itemize}

\normalsize


