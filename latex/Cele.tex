\chapter{{Cel i zakres prac}}
\label{chapter:Cel}
\thispagestyle{empty}


Cel pracy można podzielić na dwie części. Pierwszą z nich jest zrealizowanie aplikacji, która implementuje mechanizmy ekstrakcji tonów z sygnału akustycznego. Drugą częścią jest implementacja mechanizmu obsługi kontrolera MIDI. Na potrzeby tej pracy będzie wykorzystywany system operacyjny Linux i jego interfejsy programistyczne. Jednymi z podstawowych funkcji części pierwszej programu będzie wnikliwa analiza oraz przygotowanie informacji na temat pozyskanego dźwięku. Druga część aplikacji będzie rozbudowywać funkcjonalność aplikacji poprzez interpretacje danych z części pierwszej i wykorzystanie interfejsu MIDI, aby w łatwy sposób można było sterować np.: kontrolerem MIDI, syntezatorem czy innymi urządzeniami bazującymi na protokole MIDI. 


	
Wykorzystane do tego zostaną możliwie najbardziej wydajne algorytmy analizy dźwięku, które sprawdzają się przy analizie w czasie rzeczywistym. Do algorytmów tych możemy zaliczyć algorytm YIN, MPM oraz inna algorytmy korzystające z metody autokorelacji. Użyte będą również dedykowane biblioteki, które posiadają gotowe implementacji algorytmów do wykrywania tonów muzycznych.

Głównym powodem wyboru  tematu była próba połączenia zainteresowań oraz pasji z kierunkiem rozwoju zawodowego.
	
	
Kolejnym argumentem jest wytworzenie wolnego oprogramowania wychodzącego naprzeciw oczekiwaniom potencjalnych użytkowników, którzy potrzebują aplikacji cechującej się prostotą obsługi oraz niezawodnością. Aplikacja będzie dedykowana profesjonalnym muzykom, jak i amatorom, którzy potrzebują pomocy w nauce gry na instrumentach. Głównym problemem, którego rozwiązanie jest jednym z najważniejszych czynników motywacyjnych, jest udostępnienie niezawodnej aplikacji nadającej się do zastosowań przez profesjonalnych muzyków.